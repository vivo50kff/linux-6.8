% 第二章:Linux系统调度相关概念

\section{Linux系统调度相关概念} \label{sec:linux_scheduling_concepts}

本章旨在介绍支撑本项目研究的Linux内核调度理论。我们将首先概述Linux通用的调度框架,随后重点分析当前主流的完全公平调度器(CFS)的设计哲学及其在多核环境下固有的局限性,特别是其对CPU缓存亲和性的忽视。最后,我们将深入探讨缓存感知调度的核心理论,这些理论构成了我们YAT-CASCHED调度器设计的基石。

\subsection{Linux调度框架概述}

Linux内核的调度器是一个高度模块化的系统,其核心设计围绕着“调度类”(Scheduling Classes)的概念。这种设计允许不同的调度策略共存,并根据任务的特性选择最合适的策略。调度器框架的核心是`struct sched\_class`结构体,它定义了一组标准接口(函数指针),如`enqueue\_task`(将任务加入就绪队列)、`dequeue\_task`(将任务移出就绪队列)、`pick\_next\_task`(选择下一个要运行的任务)等。

系统中的调度类按照优先级顺序链接在一起。当需要选择下一个任务运行时,内核会按照从高到低的优先级遍历所有调度类,第一个成功选出任务的调度类将决定下一个在CPU上运行的进程。主要的调度类包括:

\begin{itemize}
    \item \textbf{SCHED\_STOP}: 优先级最高的调度类,用于停止CPU上的任务,通常在CPU热插拔等特殊场景使用。
    \item \textbf{SCHED\_DEADLINE}: 基于EDF(Earliest Deadline First)算法,为任务提供周期性的资源预留保证,适用于有严格截止时间要求的实时任务。
    \item \textbf{SCHED\_RR / SCHED\_FIFO}: 传统的实时调度类,基于静态优先级进行调度。
    \item \textbf{SCHED\_NORMAL (CFS)}: 完全公平调度类,是Linux系统默认的调度策略,用于普通用户进程。
    \item \textbf{SCHED\_IDLE}: 优先级最低的调度类,仅在系统没有其他任何任务可运行时才执行。
\end{itemize}

\subsection{核心调度类详解}

\subsubsection{实时调度类 (SCHED\_FIFO 和 SCHED\_RR)}
`SCHED\_FIFO`(先入先出)和`SCHED\_RR`(轮转)是Linux的传统实时调度策略。它们都基于固定的优先级(1-99),高优先级的任务总是能抢占低优先级的任务。`SCHED\_FIFO`的同优先级任务一旦获得CPU就会一直运行直到自己主动放弃或被更高优先级的任务抢占。`SCHED\_RR`则在`SCHED\_FIFO`的基础上增加了时间片机制,同优先级的任务会轮流使用CPU,避免某个任务长时间独占。这两种策略为实时应用提供了基本的时序保证,但在复杂的多核系统中,它们并未解决任务与CPU核心的绑定、缓存利用等问题。

\subsubsection{完全公平调度类 (CFS - SCHED\_NORMAL)}
CFS是当前Linux系统对普通任务的默认调度器,它的核心目标并非追求最快的响应时间,而是“完全公平”。它致力于为系统中的每个任务提供公平的CPU时间份额。

CFS通过一个巧妙的设计——虚拟运行时(virtual runtime, `vruntime`)来实现公平。每个任务都有一个`vruntime`,它记录了该任务已经获得的标准化CPU时间。`vruntime`增长得越慢,说明任务获得的CPU时间越少。CFS的调度逻辑异常简单:**永远选择当前就绪队列中`vruntime`最小的任务来运行**。

为了高效地找到`vruntime`最小的任务,CFS使用红黑树(Red-Black Tree)来组织就绪队列。红黑树的节点按`vruntime`排序,最左侧的节点即为`vruntime`最小的任务。当一个任务运行时,它的`vruntime`会随着物理时间的流逝而增加。当它被抢占或主动放弃CPU时,会根据其新的`vruntime`被重新插入红黑树。这个过程确保了那些“饥饿”的任务(`vruntime`较小)能够优先获得CPU,从而实现了长期的公平性。

\subsection{CFS的局限性:缓存亲和性问题}

尽管CFS在提供公平性方面表现出色,但其设计哲学也带来了无法忽视的副作用,尤其是在现代多核处理器架构中。这个核心问题就是**缓存亲和性(Cache Affinity)的缺失**。

缓存亲和性指的是一个任务在某个CPU核心上运行一段时间后,其使用的数据和指令会加载到该核心的各级缓存(L1, L2, L3 Cache)中。当这个任务再次被调度到同一个核心上时,它可以直接从高速缓存中读取数据,从而极大地提升执行效率。反之,如果任务被迁移到另一个CPU核心上,新核心的缓存是“冷”的,任务所需的数据必须从主内存中重新加载,这将导致显著的性能下降,这个过程称为缓存污染和缓存失效。

CFS为了在多核之间实现负载均衡(Load Balancing),会周期性地检查各个CPU核心的负载情况。如果发现负载不均,它会毫不犹豫地将任务从繁忙的核心迁移到空闲的核心,以维持其“公平”的目标。然而,这种频繁的、以负载均衡为唯一目的的迁移,恰恰破坏了任务好不容易建立起来的缓存亲和性。对于那些对缓存高度敏感的应用(如科学计算、数据处理、多媒体编解码等),CFS的这种行为会导致其执行时间(WCET)大幅波动,整体性能不升反降。这正是我们项目试图解决的核心痛点。

\subsection{缓存感知调度理论基础}

为了克服CFS的局限性,学术界和工业界提出了缓存感知调度(Cache-Aware Scheduling)的理论。其核心思想是在调度决策中明确地考虑缓存状态,旨在最大化缓存命中率,减少跨核迁移带来的性能损失。

\subsubsection{多核环境下的缓存挑战}

在多核处理器中,缓存层次结构的复杂性为调度带来了新的挑战。\cite{r4_Cache-Aware_Partitioned_Scheduler_for_Hard_Real-Time}表\ref{tab:cache-challenges}总结了主要的缓存相关问题:

\begin{table}[H]
\centering
\begin{tabular}{ccc}
\toprule
挑战类型 & 问题描述 & 影响程度 \\
\midrule
缓存亲和性 & 任务迁移导致缓存失效 & 高 \\
\midrule
缓存污染 & 不同任务间的缓存干扰 & 中等 \\
\midrule
一致性开销 & 多核间缓存同步成本 & 高 \\
\midrule
预测困难 & 缓存行为的时间不确定性 & 高 \\
\bottomrule
\end{tabular}
\caption{多核环境下的主要缓存挑战}
\label{tab:cache-challenges}
\end{table}

\subsubsection{Cache Recency Profile理论}

Cache Recency Profile (CRP)是缓存感知调度的核心理论工具,它通过建模任务间的缓存干扰来预测任务迁移对性能的影响。\cite{r6_Miss_Rate_Calculation_of_L2Cache}

\textbf{缓存距离}:定义为两次访问同一缓存块之间的唯一内存块访问数量。对于任务$T_i$中的内存访问序列,缓存距离$d_k$表示第$k$次访问与前一次访问相同内存块之间的距离。\cite{r7_Minimizing_cache_usage}

\textbf{执行时间加速比}:当任务$v_j$被分配到核心$\lambda_k$时,考虑缓存效应的执行时间加速比定义为:
$$S(v_j, \lambda_k, H, CRP) = C_j - \text{实际执行时间}$$

其中$H$表示历史分配记录,$CRP$是预测模型。

\subsubsection{缓存亲和性量化}

\textbf{缓存热度窗口}:定义一个时间窗口$\Delta t$,在此窗口内执行过的任务被认为在缓存中留有"热"数据。缓存亲和性可以通过以下公式量化:
$$\text{Affinity}(T_i, \text{core}_k) = \begin{cases}
\alpha \cdot e^{-\beta \cdot t_{\text{idle}}} & \text{if } t_{\text{idle}} \leq \Delta t \\
0 & \text{otherwise}
\end{cases}$$

其中$t_{\text{idle}}$是任务$T_i$在核心$k$上的空闲时间,$\alpha$和$\beta$是经验参数。

\subsubsection{多核调度的权衡原则}

在多核缓存感知调度中,需要平衡以下两个目标:

\begin{enumerate}
    \item \textbf{缓存局部性最大化}:$\max \sum_{i} \text{CacheHit}(T_i)$
    \item \textbf{负载均衡}:$\min \max_k \text{Load}(\text{core}_k)$
\end{enumerate}

这两个目标往往相互冲突,需要通过权重参数进行调节:
$$\text{Objective} = w_1 \cdot \text{CacheUtility} + w_2 \cdot \text{LoadBalance}$$

其中$w_1 + w_2 = 1$,权重的选择直接影响调度策略的性能特征。

\subsubsection{基本调度策略}

\textbf{最大加速优先}(Maximum Speedup First, MSF):总是将任务分配给能提供最大执行时间加速比的核心:
$$\text{core}^* = \arg\max_k S(v_j, \lambda_k, H, CRP)$$

\textbf{最小缓存影响优先}(Least Cache Impact First, LCIF):当多个核心提供相同加速比时,选择对其他任务缓存影响最小的核心。

这些理论概念构成了我们Yat-CASched调度器的设计基础。在下一章中,我们将详细介绍如何将这些理论应用到具体的系统设计