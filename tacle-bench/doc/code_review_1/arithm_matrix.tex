\documentclass[a4paper]{article}

\usepackage[english]{babel}
\usepackage[utf8]{inputenc}
\usepackage{amsmath}
\usepackage{graphicx}
\usepackage[colorinlistoftodos]{todonotes}
\usepackage{soul}

\title{Review Tackle Bench}

\author{Peter Hellinckx, Yorick De Bock}

\date{\today}

\begin{document}
\maketitle

\begin{abstract}
This document will contain the first review of the tackle bench regarding the topics.
\begin{itemize}
\item arithmetic benchmarks
\item matrix benchmarks    
\end{itemize}

\end{abstract}

\section{Arithmetic benchmarks}

\begin{itemize}
	\item  DSPstone\_fixed\_point/complex\_multiply\_fixed
	\item  \st{DSPstone\_fixed\_point/complex\_update\_fixed}
	\item  \st{DSPstone\_fixed\_point/dot\_product\_fixed}
	\item  DSPstone\_fixed\_point/n\_complex\_updates\_fixed
 	\item  DSPstone\_fixed\_point/n\_real\_updates\_fixed
	\item  \st{DSPstone\_fixed\_point/real\_update\_fixed}
	\item  DSPstone\_floating\_point/complex\_multiply\_float
	\item  \st{DSPstone\_floating\_point/complex\_update\_float}
	\item  \st{DSPstone\_floating\_point/dot\_product\_float}
	\item  DSPstone\_floating\_point/n\_complex\_updates\_float
	\item  DSPstone\_floating\_point/n\_real\_updates\_float
	\item  \st{DSPstone\_floating\_point/real\_update\_float}
	\item  MISC/ammunition
    	\begin{itemize}
            \item bits\_test (7 tests)
            \item arithm\_test (35 tests)     
        \end{itemize}
	\item  MRTC/expint
    \item  MRTC/fac
    \item  MRTC/fibcal
    \item  MRTC/prime
    \item  MRTC/qurt
    \item  \st{MRTC/sqrt}
	\item  MRTC/st
	\item  MiBench/basicmath\_small
    	\begin{itemize}
        	\item solvecubic
            \item isqrt
            \item memcpy
            \item rad2deg
            \item deg2rad
            \item conversions
            
        \end{itemize}
    	
\end{itemize}

\subsection{DSPstone\_fixed\_point/complex\_multiply\_fixed}
	TO BE DISCUSSED: I would keep this one to test the elementary multiplication of a complex number. Then again it strongly resembles a D2 matrix mult (DSPstone\_fixed\_point/matrix1\_fixed).
    
\subsection{DSPstone\_fixed\_point/complex\_update\_fixed}
	REMOVE: Can be removed as it is part of DSPstone\_fixed\_point/n\_complex\_updates\_fixed
    
\subsection{DSPstone\_fixed\_point/dot\_product\_fixed}
	REMOVE: Can be removed as it is part of DSPstone\_fixed\_point/matrix1\_fixed

\subsection{DSPstone\_fixed\_point/n\_complex\_updates\_fixed}
	KEEP: This benchmark can be configured to include DSPstone\_fixed\_point/complex\_update\_fixed 

\subsection{DSPstone\_fixed\_point/n\_real\_updates\_fixed}
	KEEP: This benchmark can be configured to include DSPstone\_fixed\_point/real\_update\_fixed
    
\subsection{DSPstone\_fixed\_point/real\_update\_fixed}
	REMOVE: Can be removed as it is part of DSPstone\_fixed\_point/n\_real\_updates\_fixed
    

\subsection{DSPstone\_floating\_point/complex\_multiply\_floating}
	TO BE DISCUSSED: I would keep this one to test the elementary multiplication of a complex number. Then again it strongly resembles a D2 matrix mult (DSPstone\_fixed\_point/matrix1\_fixed).
    
\subsection{DSPstone\_floating\_point/complex\_update\_floating}
	REMOVE: Can be removed as it is part of DSPstone\_floating\_point/n\_complex\_updates\_floating
    
\subsection{DSPstone\_floating\_point/dot\_product\_floating}
	REMOVE: Can be removed as it is part of DSPstone\_floating\_point/matrix1\_floating

\subsection{DSPstone\_floating\_point/n\_complex\_updates\_floating}
	KEEP: This benchmark can be configured to include DSPstone\_floating\_point/complex\_update\_floating 

\subsection{DSPstone\_floating\_point/n\_real\_update\_floating}
	KEEP: This benchmark can be configured to include  DSPstone\_floating\_point/real\_update
    
\subsection{DSPstone\_floating\_point/real\_update\_floating}
	REMOVE: Can be removed as it is part of DSPstone\_floating\_point/n\_real\_updates\_floating

    
\subsection{MISC/ammunition}   
	KEEP: Integer library addressing overflow. It contains 7 bits (mem) tests and 35 useful arithmetic tests (overflow). It can probably be scaled down but not removed.
    
	\subsubsection{bits}
    	KEEP:Unique benchmark on mem actions
    \subsubsection{arithm}
    	KEEP:Unique benchmark on overflow
        
\subsection{MRTC/expint}
	KEEP: Unique. Update by Martin: removed 26 April 2016 as license does not fit.
\subsection{MRTC/fac}
	KEEP: Unique
\subsection{MRTC/fibcal}
	KEEP: Unique
\subsection{MRTC/prime}
	KEEP:Unique
\subsection{MRTC/qurt}
	KEEP: Unique
\subsection{MRTC/sqrt}
	REMOVE: Code available in MRTC/qurt and MRTC/st
\subsection{MRTC/st}
	KEEP: Unique
\subsection{MiBench/basicmath\_small}
    \subsubsection{solvecubic}
    	KEEP:Unique
    \subsubsection{isqrt}
    	KEEP:Unique
    \subsubsection{memcpy}
    	KEEP: necessary for other benchmarks
    \subsubsection{rad2deg}
    	KEEP: unique
    \subsubsection{deg2rad}
    	KEEP: Unique
    \subsubsection{conversions}
    	KEEP: Unique
    

\section{Matrix benchmarks}

\begin{itemize}
	\item \st{DSPstone\_fixed\_point/matrix1x3\_fixed}
	\item DSPstone\_fixed\_point/matrix1\_fixed
	\item \st{DSPstone\_floating\_point/matrix1x3\_float}
	\item DSPstone\_floating\_point/matrix1\_float
	\item MRTC/countnegative
    \item MRTC/ludcmp
    \item MRTC/matmult
    \item MRTC/minver
\end{itemize}

  \subsection{DSPstone\_fixed\_point/matrix1x3\_fixed}
      REMOVE: Can be removed as it is part of DSPstone\_fixed\_point/matrix1\_fixed
    
    
  \subsection{DSPstone\_fixed\_point/matrix1\_fixed}
      KEEP: This benchmark can be configured to include DSPstone\_fixed\_point/matrix1x3\_fixed and DSPstone\_fixed\_point/dot\_product\_fixed


  \subsection{DSPstone\_floating\_point/matrix1x3\_float}
  REMOVE: Can be removed as it is part of DSPstone\_float\_point/matrix1\_float
  \subsection{DSPstone\_floating\_point/matrix1\_float}
  KEEP: This benchmark can be configured to include DSPstone\_float\_point/matrix1x3\_float and DSPstone\_float\_point/dot\_product\_float
  
  \subsection{MRTC/countnegative}
  	KEEP: Unique function and matrix is 2 dimensional array
  \subsection{MRTC/ludcmp}
  	KEEP: Unique function and matrix is 2 dimensional array
  \subsection{MRTC/matmult}
  	KEEP: Resembles DSPstone\_fixed\_point/matrix1\_fixed BUT array is 2 dimensional in this case
  \subsection{MRTC/minver}
  	KEEP: Floating point matrix inversion


\end{document}