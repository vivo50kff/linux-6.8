\documentclass[a4paper]{article}

\usepackage[english]{babel}
\usepackage[utf8]{inputenc}
\usepackage{amsmath}
\usepackage{graphicx}
\usepackage[colorinlistoftodos]{todonotes}
\usepackage{soul}

\title{Review Tackle Bench}

\author{Jörg Mische}

\date{\today}

\begin{document}
\maketitle

\begin{abstract}
This document will contain the first review of the tackle bench regarding the 
topic DSP filters.
\end{abstract}


\section{Integer DSP Filters}

\begin{itemize}
	\item \st{SPstone\_fixed\_point/convolution\_fixed}
	\item \st{DSPstone\_fixed\_point/fft\_16\_7}
	\item \st{DSPstone\_fixed\_point/fft\_16\_13}
	\item \st{DSPstone\_fixed\_point/fft\_1024\_7}
	\item DSPstone\_fixed\_point/fft\_1024\_13
	\item DSPstone\_fixed\_point/iir\_biquad\_N\_sections\_fixed
	\item \st{DSPstone\_fixed\_point/iir\_biquad\_one\_section\_fixed}
	\item DSPstone\_fixed\_point/lms\_fixed
	\item MRTC/edn
	\item MRTC/fdct
	\item MRTC/fft1
	\item MRTC/fir
	\item MRTC/fjdctint
	\item MRTC/lms
\end{itemize}



\subsection{DSPstone\_fixed\_point/convolution\_fixed}
	REMOVE: Extremely simple loop, part of FIR-filters
\subsection{DSPstone\_fixed\_point/fft\_16\_7}
	REMOVE: same algorithm as DSPstone\_fixed\_point/fft\_16\_13, only one loop is missing
\subsection{DSPstone\_fixed\_point/fft\_16\_13}
	REMOVE: same algorithm as DSPstone\_fixed\_point/fft\_16\_7, one additional loop and more bits per entry
\subsection{DSPstone\_fixed\_point/fft\_1024\_7}
	REMOVE: same algorithm as DSPstone\_fixed\_point/fft\_16\_7, but larger matrix
\subsection{DSPstone\_fixed\_point/fft\_1024\_13}
	KEEP: same algorithm, but largest workload

	TO BE DISCUSSED: Alternatively keep smaller workload?
\subsection{DSPstone\_fixed\_point/fir2dim\_fixed}
	KEEP: two dimensional FIR
\subsection{DSPstone\_fixed\_point/fir\_fixed}
	REMOVE: very simple, part of DSPstone\_fixed\_point/lms\_fixed, MRTC/fir is more complex
\subsection{DSPstone\_fixed\_point/iir\_biquad\_N\_sections\_fixed}
	KEEP
\subsection{DSPstone\_fixed\_point/iir\_biquad\_one\_section\_fixed}
	REMOVE: no loops, one iteration of DSPstone\_fixed\_point/iir\_biquad\_N\_sections\_fixed
\subsection{DSPstone\_fixed\_point/lms\_fixed}
	REMOVE: very simple, MRTC/lms is more complex
\subsection{MRTC/edn}
	KEEP: several filters (including DCT, FIR and IIR)
\subsection{MRTC/fdct}
	TO BE DISCUSSED: very similar to MRTC/fjdctint, strange code parts here (multiplications with constants)
\subsection{MRTC/fft1}
	KEEP: Different implementation from DSPstone (other loops, arrays instead of pointers)
\subsection{MRTC/fir}
	KEEP: One dimensional FIR, more complex than DSPstone\_fixed\_point/fir\_fixed
\subsection{MRTC/fjdctint}
	KEEP: Discrete Cosine Transform, MRTC/fdct is very similar
\subsection{MRTC/lms}
	KEEP: Least Mean Square approximation




\section{Floating Point DSP Filters}

\begin{itemize}
	\item DSPstone\_floating\_point/convolution\_float
	\item DSPstone\_floating\_point/fir2dim\_float
	\item DSPstone\_floating\_point/fir\_float
	\item DSPstone\_floating\_point/iir\_biquad\_N\_sections\_float
	\item DSPstone\_floating\_point/iir\_biquad\_one\_section\_float
	\item DSPstone\_floating\_point/lms\_float
	\item StreamIt/filterbank
\end{itemize}
	TO BE DISCUSSED: The DSPstone\_floating\_point programs are identical to
	the fixed variants, the only difference is a ``\#define TYPE float''.
	Shall we keep both versions or provide a mechanism to switch between
	integer and floating point? The floating point versions are the newer
	ones, therefore these should be taken.

\subsection{StreamIt/filterbank}
	KEEP: Real floating point benchmark

    

\end{document}