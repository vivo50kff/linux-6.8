% 第九章:总结

\section{总结与展望} \label{sec:conclusion}

本项目初步探索了面向多核实时系统的轻量化缓存感知型调度器——Yat-CAShed的设计与实现。通过对Linux内核调度子系统的研究,我们尝试实现了一个新的调度类,其核心思路是利用CPU缓存亲和性来优化任务布局,以期减少因任务迁移产生的性能开销,为特定应用场景提供更稳定的执行延迟。

在项目的初赛阶段,我们从理论学习到动手实践,克服了一系列挑战。我们系统性地学习了实时系统调度理论和Linux内核机制,攻克了内核模块化编程、\texttt{sched\_class}接口集成等技术难题,并在QEMU虚拟环境中进行了细致的调试,解决了开发过程中遇到的诸多问题,最终使调度器原型得以基本运行。

目前,Yat-CAShed调度器完成了核心功能的初步开发与验证。在搭建的测试环境中,初步结果显示,我们的设计能够在一定程度上提升任务的缓存局部性,并在特定负载下,相较于标准CFS调度器展现出了一定的延迟优势。这初步验证了以缓存亲和性为核心的调度策略在特定场景下的可行性。

同时,我们清醒地认识到,当前的工作仅是万里长征的第一步。调度器原型在功能的完备性、性能的普适性以及代码的健壮性上仍有较大提升空间。例如,在关键任务抢占、对异构计算架构(如ARM big.LITTLE)的感知、NUMA架构的深度支持以及能耗优化等方面,我们尚未进行深入探索。这些不仅是我们未来工作的方向,也为下一阶段的开发明确了具体目标。

总而言之,Yat-CAShed项目是团队一次宝贵的内核编程实践,也是对操作系统底层调度机制的一次有益探索。它为解决特定实时场景下的低延迟需求提供了一个具备潜力的设计思路,并为我们继续深入研究更智能、更高效的调度策略打下了实践基础。