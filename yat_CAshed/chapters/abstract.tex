% 摘要章节

\section*{摘要}

针对OSproj59赛题,CFS调度器存在臃肿、缺乏关键线程优先调度、时延不可控等问题。本项目基于缓存感知调度理论,设计Yat-CASched调度器。首先开发基于Java的仿真平台验证Cache Recency Profile (CRP)理论的有效性。考虑初赛时间限制,内核实现采用简化策略:基于10ms热度窗口,优先将任务分配给其上次运行的CPU核心,减少跨核迁移;超时后允许负载均衡。该方案优势:(1)轻量化设计,降低调度开销;(2)减少任务迁移,降低调度延迟;(3)改善缓存利用率,提升执行效率。项目采用"理论学习→Java仿真验证→简化算法设计→内核实现"技术路线,成功将SCHED\_YAT\_CASCHED策略集成到Linux 6.8内核。测试表明,相比CFS能有效减少跨核迁移,提升缓存利用率,符合赛题"轻量化"和"低延迟"要求。本项目验证了通过缓存优化解决调度器臃肿问题的可行性,为后续关键进程调度优化奠定基础。

\textbf{关键词:}缓存感知型任务调度、Linux内核调度技术、性能优化、仿真平台搭建