% 第一章:项目介绍


\section{项目介绍} \label{sec:intro}

\subsection{项目背景和意义}

\subsubsection{多核缓存感知调度的必要性}

随着多核处理器技术的发展,现有Linux调度器在特定场景下的局限性日益凸显。当前Linux内核中的完全公平调度器(CFS)主要关注任务公平性和负载均衡,但在设计时并未充分考虑现代多核处理器的缓存层次结构特征。


现有调度器存在三个核心问题:调度器结构日益臃肿,代码复杂度持续上升,调度开销增加;缺乏对关键线程的优先调度支持,无法有效识别和处理重要任务;时延不可控,在高并发场景下任务调度延迟难以预测,严重影响系统响应性能。\cite{r5_Fair_scheduling_algorithm}

\subsubsection{缓存感知调度的技术价值}

现代多核处理器采用了复杂的缓存层次结构来缓解处理器与内存之间的速度差异。\cite{r2_A_shared_cache-aware_Task_scheduling_strategy}表\ref{tab:cache-hierarchy}详细展示了典型多核处理器的缓存配置:

\begin{table}[h]
\centering
\begin{tabular}{ccccc}
\toprule
缓存级别 & 容量 & 访问延迟 & 共享范围 & 主要特征 \\
\midrule
L1缓存 & 32KB & 1-2个时钟周期 & 每核心独有 & 最快,容量最小 \\
L2缓存 & 256KB & 10-20个时钟周期 & 每核心独有 & 速度快,私有缓存 \\
L3缓存 & 8-32MB & 30-40个时钟周期 & 多核心共享 & 容量大,共享缓存 \\
主内存 & 8GB+ & 100-300个时钟周期 & 全系统共享 & 容量最大,速度最慢 \\
\bottomrule
\end{tabular}
\caption{典型多核处理器缓存层次结构对比}
\label{tab:cache-hierarchy}
\end{table}

这种设计充分利用了程序的时间局部性和空间局部性特征。然而,当任务发生跨核迁移时,这种精心设计的缓存层次结构的优势被严重破坏。表\ref{tab:migration-impact}展示了任务迁移对不同级别缓存的具体影响:

\begin{table}[h]
\centering
\begin{tabular}{|c|c|c|c|}
\hline
迁移场景 & L1/L2缓存影响 & L3缓存影响 & 性能损失估计 \\
\hline
同核心内调度 & 无影响 & 无影响 & 0\% \\
相邻核心迁移 & 完全失效 & 部分保留 & 10-15\% \\
跨NUMA节点迁移 & 完全失效 & 完全失效 & 20-30\% \\
高频迁移 & 持续失效 & 持续污染 & 35\%以上 \\
\hline
\end{tabular}
\caption{任务迁移对缓存性能的影响分析}
\label{tab:migration-impact}
\end{table}

更严重的是,这种性能损失往往是累积性的。在高负载的多任务环境中,频繁的任务迁移会导致整个系统的缓存效率持续下降,形成性能恶化的连锁反应。据相关研究表明,在某些应用场景下,不合理的任务迁移可能导致20\%以上的性能损失。

\subsubsection{项目的理论和实践意义}

设计和实现缓存感知调度器对我们团队具有重要的学习价值和实践意义。从学习角度来看,这项工作让我们深入理解缓存感知调度理论,通过实际的仿真验证和内核实现,加深了对操作系统调度机制和现代处理器架构的理解。

从工程实践的角度来看,缓存感知调度器能够在特定应用场景下提供一定的性能提升。在高性能计算、数据库系统等缓存敏感的应用中,合理的缓存感知调度能够减少任务迁移带来的性能损失。虽然我们的实现相对简单,但大规模仿真验证显示了缓存感知调度的潜在价值。

特别值得关注的是,本项目采用的轻量化设计理念体现了工程实现中的实用性考虑。通过10ms热度窗口的简化策略,我们在保证基本缓存感知效果的同时,避免了过度复杂的实现,使得该调度器能够在保证稳定性的前提下获得一定的性能改进。

此外,完整的设计和实现过程也为我们提供了宝贵的系统级开发经验。从理论学习到仿真验证,再到内核实现,这个完整的技术路线让我们对操作系统内核开发有了更深入的认识和实践体验。

\subsection{项目目标和分析}



\subsubsection{总体目标解读}

\textbf{目标1:设计并实现轻量化的缓存感知调度器}

目标解读:需要从理论研究出发,深入理解缓存感知调度的核心原理,特别是Cache Recency Profile (CRP)模型的数学基础和物理意义。在此基础上,设计一个既体现理论先进性又具备工程可行性的调度算法。调度器必须针对多核环境下的缓存局部性进行优化,同时保持轻量化的设计原则,避免过度复杂的实现。我们需要明确选择基于缓存热度窗口的简化策略,并确定该策略下调度器需要优化的核心性能指标,如任务迁移频率、缓存命中率、调度延迟等。

\textbf{目标2:验证调度器在多种场景下的性能优势}

目标解读:这需要我们设计并执行一系列全面的实验,将所设计的调度器与现有的调度策略进行系统性对比。首先通过大规模仿真平台进行大规模的理论验证,涵盖不同工作负载特征和系统利用率水平。然后在Linux内核中进行实际的性能测试,与CFS等现有调度器进行对比分析。性能评估的指标需要多维度覆盖,包括但不限于:缓存命中率、任务完成时间、系统响应延迟、CPU利用率、任务迁移次数、系统吞吐量等。

\textbf{目标3:提供完整的技术方案和实现指南}

目标解读:需要撰写一份详细的技术文档,全面阐述缓存感知调度器的设计思想、理论基础、算法实现以及性能优化策略。这个文档应该详细记录从理论研究到工程实现的完整过程和关键决策。同时提供完整的实现指南,详细描述如何将该调度器集成到Linux内核中,包括必要的代码修改、编译配置和部署步骤。最后,编写详细的性能评估报告,全面描述实验方法、数据收集过程、分析结果以及结论,为后续的研究和应用提供可靠的参考。

\subsubsection{技术挑战分析}

在技术实现方面,项目面临多个层次的挑战。表\ref{tab:tech-challenges}总结了主要技术挑战及其对应的解决策略:

\begin{table}[h]
\centering
\begin{tabular}{|c|c|c|c|}
\hline
挑战类别 & 具体难点 & 复杂度 & 解决策略 \\
\hline
理论理解 & CRP模型数学原理 & 高 & 文献研读+仿真验证 \\
仿真建模 & 多核缓存系统建模 & 高 & 大规模仿真平台+精确参数 \\
算法简化 & 理论到工程的转化 & 中 & 10ms热度窗口策略 \\
内核集成 & Linux调度框架集成 & 高 & 遵循内核编程规范 \\
性能平衡 & 缓存vs负载均衡权衡 & 中 & 动态权衡机制 \\
\hline
\end{tabular}
\caption{项目技术挑战及解决策略}
\label{tab:tech-challenges}
\end{table}

算法简化是关键挑战,如何在保证缓存感知效果的前提下大幅简化理论模型的复杂度,使其适合在生产环境的Linux内核中实际部署,这需要在理论完整性、工程可行性和性能优化之间找到最佳平衡点。

\subsubsection{解决方案设计}

针对上述挑战,我们制定了系统性的解决方案。在理论研究方面,通过深入研读相关文献并开发大规模仿真器来验证CRP理论的有效性,确保理论基础的扎实性。在算法简化方面,采用基于10ms缓存热度窗口的策略,这个时间窗口既考虑了典型应用的缓存访问模式,又保持了实现的简洁性。

在内核实现方面,设计完整的SCHED\_YAT\_CASCHED调度类,严格遵循Linux内核的编程规范和接口要求。在性能优化方面,实现缓存感知和负载均衡的智能权衡机制,通过维护任务的缓存热度状态和CPU负载信息,动态决策任务分配策略。

\subsection{项目要求实现内容}

根据OSproj59赛题的具体要求,本项目需要完成一系列技术实现任务。

\subsubsection{赛题基本要求}

赛题明确提出了五个核心要求。调度器轻量化要求我们实现一个结构简洁、运行开销低的调度算法,避免传统调度器复杂度过高的问题。关键线程优先调度要求支持对重要任务的优先处理机制,能够识别和区分不同任务的重要程度。时延可控要求降低任务调度和切换的时间开销,提供更加可预测的系统响应时间。缓存感知要求充分考虑处理器缓存特性来优化任务分配,这是本项目的核心特色。多核适配要求充分利用多核处理器的并行处理能力,确保调度策略在多核环境下的有效性。

\subsubsection{技术实现要求}

在理论研究方面,我们需要深入学习缓存感知调度的相关理论文献,特别是要理解Cache Recency Profile模型的数学原理和物理意义。同时要分析现有调度器如CFS的优缺点和改进空间,为后续的算法设计提供理论基础。

在仿真验证方面,开发大规模多核缓存调度模拟器是关键任务。这个模拟器需要实现CRP理论模型的完整仿真,能够准确模拟多级缓存的行为和性能特征。通过与传统调度算法的对比,验证新算法在不同工作负载下的有效性,为后续的内核实现提供数据支撑。

在内核实现方面,需要设计一个轻量化的缓存感知调度算法,实现名为SCHED\_YAT\_CASCHED的调度策略。这个策略必须能够集成到Linux 6.8内核调度框架中,并确保与现有调度器的兼容性。算法的核心是维护任务的缓存热度信息,并基于这些信息做出调度决策。

在测试验证方面,需要设计综合性能测试方案,重点测试调度延迟、缓存利用率等关键指标。通过在不同应用场景下的性能测试,验证调度器的实际效果,并与CFS调度器进行详细的对比分析。

\subsection{项目预期实现成果}

经过深入的需求分析和技术调研,我们对项目的预期成果有了清晰的规划。

\subsubsection{理论研究成果}

在理论研究方面,我们完成了三个主要成果。首先是对Cache Recency Profile理论模型的深入学习,通过文献调研和算法分析,我们充分掌握了CRP模型的核心思想和基本原理,为后续的算法设计提供了理论指导。其次是通过大规模仿真验证建立的性能评估框架,这个框架为算法性能测试和结果分析提供了实用的评价工具。最后是基于缓存热度窗口的轻量化调度算法设计方案,这个方案结合了理论研究和仿真验证的成果,提出了一个既有理论支撑又具备实践可行性的简化解决方案。

\subsubsection{软件实现成果}

软件成果主要包括两个核心系统。第一个是大规模仿真平台,这是一个完整的多核缓存调度仿真系统。该平台支持Yat-CASched算法和传统WFD算法的对比分析,能够提供包括任务完成时间、缓存命中率、负载均衡度、能耗等多维度性能指标的统计分析。同时,我们还开发了配套的可视化分析工具,能够自动生成专业级的性能对比图表。

第二个是Linux内核调度器,即SCHED\_YAT\_CASCHED调度策略的完整实现。这个调度器已经成功集成到Linux 6.8内核中,支持10ms缓存热度时间窗口的精确控制,能够实现缓存亲和性和负载均衡的动态权衡。整个实现遵循Linux内核的编程规范和接口要求,确保了系统的稳定性和兼容性。

\subsubsection{性能验证成果}

通过大规模仿真验证,我们已经获得了令人鼓舞的性能改进数据。表\ref{tab:simulation-results}详细展示了仿真实验的核心成果:

\begin{table}[h]
\centering
\begin{tabular}{cccc}
\toprule
性能指标 & Yat-CASched算法 & WFD基准算法 & 改进幅度 \\
\midrule
缓存命中率 & 75.2\% & 50.0\% & +50.4\% \\
\midrule
算法胜率 & 94.9\% & 5.1\% & +94.9\% \\
\midrule
系统能耗 & 617.04 & 684.99 & -9.92\% \\
\midrule
测试案例数 & \multicolumn{3}{c}{98个综合测试案例} \\
\bottomrule
\end{tabular}
\caption{大规模仿真验证核心性能指标对比}
\label{tab:simulation-results}
\end{table}

在内核实现方面,我们预期通过实际测试验证以下效果(下一步工作),如表\ref{tab:kernel-expectations}所示:

\begin{table}[h]
\centering
\begin{tabular}{ccc}
\toprule
性能维度 & 预期改进 & 验证方法 \\
\midrule
调度延迟 & 相比CFS降低10-15\% & 微基准测试 \\
\midrule
缓存利用率 & 减少跨核迁移30\% & 性能计数器监控 \\
\midrule
系统稳定性 & 保持负载均衡特性 & 长时间压力测试 \\
\midrule
响应时间 & 改善实时任务响应 & 延迟敏感应用测试 \\
\bottomrule
\end{tabular}
\caption{Linux内核实现预期性能改进}
\label{tab:kernel-expectations}
\end{table}

\subsubsection{文档交付成果}

项目将产生完整的文档体系,包括这份详细的技术报告,记录了项目的完整设计思路和实现过程。同时还有详细的实验报告,提供了全面的性能测试和分析结果。为了便于用户使用,我们还准备了调度器使用和配置指南。最后,所有的源码都配备了完整的注释和API文档,便于后续的维护和扩展。

\subsection{项目目前完成情况}

截至目前,项目已按照既定技术路线完成了主要阶段的工作:

\subsubsection{理论学习阶段(已完成)}

\begin{itemize}
    \item \textbf{理论文献调研}:深入研究了缓存感知调度相关的理论文献
    \item \textbf{CRP模型理解}:完成了Cache Recency Profile理论模型的学习和理解
    \item \textbf{现有调度器分析}:分析了Linux CFS调度器的实现机制和性能瓶颈
    \item \textbf{技术方案设计}:确定了基于缓存热度窗口的简化实现策略
\end{itemize}

\subsubsection{大规模仿真验证阶段(已完成)}

\begin{itemize}
    \item \textbf{仿真平台开发}:完成了完整的多核缓存调度仿真系统
    \begin{itemize}
        \item 实现了Yat-CASched Resource Variable算法
        \item 实现了Worst Fit Decreasing基准算法
        \item 建立了多级缓存层次结构模型
        \item 集成了UUnifastDiscard任务生成器
    \end{itemize}
    \item \textbf{大规模实验验证}:完成了98个测试案例的对比实验
    \begin{itemize}
        \item 相比传统的WFD,Yat-CASched调度器在94.9\%的测试案例中更早结束
        \item 缓存命中率提升50.4\%
        \item 系统能耗降低9.92\%
    \end{itemize}
    
\end{itemize}

\subsubsection{内核初步实现阶段(已完成)}

\begin{itemize}
    \item \textbf{调度策略实现}:成功实现了SCHED\_YAT\_CASCHED调度类
    \begin{itemize}
        \item 调度策略ID设置为8
        \item 实现了10ms缓存热度时间窗口控制
        \item 支持缓存亲和性和负载均衡的动态权衡
    \end{itemize}
    \item \textbf{内核集成}:完成了与Linux 6.8内核的集成
    \begin{itemize}
        \item 修改了核心调度器文件
        \item 实现了调度策略注册和初始化
        \item 集成了任务创建和调度逻辑
    \end{itemize}
    \item \textbf{测试验证}:完成了基本功能测试
    \begin{itemize}
        \item 验证了调度器的正常工作
        \item 测试了缓存亲和性效果
        \item 确认了系统稳定性
    \end{itemize}
\end{itemize}

\subsubsection{当前工作状态}

\begin{itemize}
    \item \textbf{核心功能}:✅ 已完成 - 调度器核心功能已实现并验证
    \item \textbf{性能测试}:✅ 已完成 - 大规模仿真已验证算法有效性
    \item \textbf{文档整理}:�� 进行中 - 正在完善技术报告和用户文档
    \item \textbf{代码优化}:�� 进行中 - 持续优化代码质量和注释
\end{itemize}

\subsection{初赛项目开发历程}

本项目的开发历程体现了一个完整的从理论到实践的技术路线。表\ref{tab:development-timeline}展示了项目开发的详细时间安排:

\begin{table}[h]
\centering
\begin{tabular}{cc}
\toprule
预期内容 & 预期时间 \\
\midrule
前期准备工作:组队、联系项目导师,往届参赛学长介绍参赛经验 & 4.19-4.25 \\
\midrule
项目初步调研:可行性、实现方向、项目框架、技术栈,确定选题 & 4.26-5.2 \\
\midrule
项目深度调研:复现已有项目代码、论文,初步弄懂实现逻辑,思考改进方 & \\
式,深入调研Linux现有实时调度框架 & 5.3-5.9 \\
\midrule
进行了第一次项目知识研讨,首次添加项目文件,标准化项目文档 & 5.10-5.16 \\
\midrule
初步完成代码框架设计和试编译 & 5.17-5.23 \\
\midrule
仓库建设,初赛资料准备,撰写初赛文档 & 5.24-5.30 \\
\bottomrule
\end{tabular}
\caption{项目开发历程时间线}
\label{tab:development-timeline}
\end{table}

整个开发过程严格按照时间规划执行,从4月19日的前期准备工作开始,到5月30日完成初赛文档撰写。前期准备阶段我们完成了团队组建和导师联系,充分利用了往届学长的参赛经验指导。

项目调研阶段分为初步调研和深度调研两个环节,我们深入研读了Fedorova等人关于Cache Recency Profile的开创性工作,复现了相关项目代码,并深入调研了Linux现有实时调度框架,为后续实现奠定了坚实基础。

开发实施阶段我们按计划进行了项目知识研讨,建立了标准化的项目文档体系,完成了代码框架设计和试编译工作。最后的仓库建设和文档撰写阶段确保了项目的完整交付和规范性。

\subsection{队伍简介} \label{sec:team-intro}
我们的队伍名为“从容应队”,三位成员和校内指导老师均来自于中山大学
计算机学院,队员均为本科二年级学生,有比赛相关的专业经验和经历,现将队伍成员和指导老师简要介绍如下。
\subsubsection{队伍成员简介}
\begin{itemize}
\item 队长\textbf{林炜东},中山大学计算机学院计算机科学与技术专业 2023
级本科生,2023年进入赵帅老师实验室中工作。
\item 队员\textbf{马福泉},中山大学计算机学院计算机科学与技术专业 2023
级本科生,2023年进入赵帅老师实验室中工作。
\item 队员 \textbf{刘昊},中山大学计算机学院计算机科学与技术专业 2023
级本科生,2023年进入赵帅老师实验室中工作。
\end{itemize}
% \subsubsection{项目导师简介}
% 谢秀奇,华为公司内核技术专家,担任openEuler社区技术委员会委员,参与了多项openEuler内核相关工作。

\subsubsection{指导老师简介}
\begin{itemize}
\item 指导老师\textbf{赵帅},硕士生导师,“百人计划”引进副教授。本科(2008.09‐2012.07)毕
业于西安工业大学,硕士(2013.10‐2014.10)、博士(2014.10‐2018.08)毕业于英国
约克大学。2018 年 10 月至 2022 年 9 月为英国约克大学计算机学院博士后。主要
从事实时系统、操作系统领域的理论与应用研究,旨在围绕操作系统,为上层应用
提供高性能、硬实时的计算与通讯保障,专注于复杂实时系统设计与分析、系统资
源共享与管理、硬软件协同设计与寻优等具体方向。相关工作发表在 DAC、RTSS
与 IEEE Trans. TDPS、TC、TCAD 等国际顶级会议与重要期刊,获得约克大学计算
机学院海外研究生奖与三次最佳论文提名。

\item 校外导师\textbf{谢秀奇}(华为技术有限公司)。
\item  校外导师\textbf{成坚}(华为技术有限公司)。
\end{itemize}


\subsection{初赛项目分工}

\begin{table}[H]
\centering
\begin{tabular}{p{2.5cm} p{11cm}}
\toprule
\textbf{小组成员} & \textbf{分工内容} \\
\midrule
林炜东 &
\begin{itemize}
    \item 负责项目整体架构设计与技术路线规划,主导理论研究与文档撰写
    \item 深入调研多核缓存感知调度理论,主导Cache Recency Profile等核心理论学习
    \item 设计并实现Yat-CASched调度器的核心算法与系统架构
    \item 负责Linux内核调度器的具体实现与集成,完成SCHED\_YAT\_CASCHED调度类开发
    \item 负责内核代码的调试、测试与优化,确保调度器在多核环境下的稳定性
    \item 统筹团队进度,协调各成员任务分配
\end{itemize}
\\
\midrule
马福泉 &
\begin{itemize}
    \item 负责仿真平台的开发与大规模实验验证,性能数据分析与可视化
    \item 参与仿真平台的数据结构设计与实验脚本开发
    \item 参与调度算法的工程简化与性能优化,设计缓存热度窗口等关键机制
    \item 负责内核性能测试、基准对比与数据收集
    \item 参与技术文档与用户手册的编写
\end{itemize}
\\
\midrule
刘昊 &
\begin{itemize}
    \item 负责实时系统与多核调度相关理论调研,梳理任务模型与调度算法分类
    \item 负责实验数据的统计分析与可视化展示
    \item 参与项目文档、实验报告和使用手册的撰写
    \item 协助团队进行代码测试与文档校对
\end{itemize}
\\
\bottomrule
\end{tabular}
\caption{队伍成员分工表}
\label{tab:team-division}
\end{table}